\section{データセットと手法\label{methodology}}

\subsection{データセット}
データセットとなるOSSは,GitHubの検索用URLを用いたスクレイピングで,2022年12月24日から25日にかけて収集した.
以下の条件のもと,スター数の多い順に2,000件のレポジトリを選択している.
\begin{itemize}
    \item[(1) ] レポジトリの言語がJavaである.
    \item[(2) ] 最終プッシュが2020年3月14日以降である.
\end{itemize}
(2)の2020年3月14日は,レコードがプレビュー機能として実装されたJava14のリリース日である.
この条件は,レコードが含まれている可能性の低いレポジトリを除外する目的で設定している.

\subsection{手法}
各RQにおける解析は,レポジトリ中の.javaファイルから抽象構文木を生成して行った.
抽象構文木の生成に用いたのは,Java17APIのパッケージcom.sun.source.treeのインターフェース群をベースにして制作した独自のパーサである.
独自のパーサを用いた理由は,レコードがJava16以降の言語仕様であり,対応するバージョンがJava15までであるJavaParserなどでは解析できない可能性を考慮したからである.

\subsubsection{RQ1\label{rq1_method}}
RQ1については,はじめにレポジトリをクローンした時点(2022年12月25日)でのレコードの利用数を集計した.
それからデータセット中の各レポジトリを,2020年4月から2022年12月の期間で各月1日時点のコミットに巻き戻し,レコードの利用数を集計した.

\subsubsection{RQ2\label{rq2_method}}
RQ1の結果,71件のレポジトリでレコードの使用履歴がみられた.
よってRQ2では,これら71件のレポジトリに含まれるレコードに対して,ヘッダ,実装インタフェースの側面から分析を行った.
さらに同じレポジトリ群に含まれるクラスについても,インスタンスフィールド,実装インタフェースの側面から分析を行い,結果を比較した.
なお,フィールド型およびインタフェースは,アノテーションを除去した文字列ベースで分類している.

\subsubsection{RQ3\label{rq3_method}}
RQ3では,レコードの追加・削除について統計を取るため,各レポジトリに対して次の操作を行った.
\begin{itemize}
    \item[(1) ] git diffコマンドで,親コミットから変更されたファイルのパスを取得する.
    \item[(2) ] (1)のファイルが存在する場合,ファイルに含まれる型のクラスパスと型の情報を取得する.存在しない場合,削除ファイルとして記録する.
    \item[(3) ] 親コミットに遡上する.
    \item[(4) ] (1)のファイルが存在する場合,型のクラスパスと型の情報を取得する.存在しない場合,追加ファイルとして記録する.
    \item[(5) ] (2)と(4)の情報を比較する.
\end{itemize}
上記の操作を繰り返していき,レコードの追加および削除の件数を集計した.
さらに,クラスからレコードの変更については,コンストラクタ,ゲッタメソッド,toString/equals/hashCodeメソッド,その他メソッドの増減についても集計した.
なお,ここでいうゲッタメソッドは,名前がインスタンスフィールド名,あるいはインスタンスフィールド名に接頭辞getを付けたのものであり,引数のないメソッドとしている.
また,toString/equals/hashCodeの各メソッドはシグネチャで判定している.

\subsubsection{RQ4\label{rq4_method}}
RQ3を実施する過程で,クラスからレコードへのリファクタリングを一度に50件以上行うコミットを複数発見した.
RQ4における1つ目の調査では,これらのコミットを目視調査することで,リファクタリング作業に伴うコストについて紐解いていく.
また,同じくRQ3の結果より,レコードからクラスに変更されたケースが31件存在することがわかった.
これらの変更は,もし意図的なものならば,レコードの利用で何かしらの弊害があったから行われたと考えることができる.
そしてレコードの利用による弊害は,開発者がレコードの導入を渋る理由の一端を担っている可能性があり,今後レコードを用いたリファクタリングを設計する上で重要な要素になると考えられる.
よってRQ4における2つ目の調査では,レコードからクラスへの変更についてソースコードの差分やコメント,GitHub上の議論を目視し,理由を調査した.