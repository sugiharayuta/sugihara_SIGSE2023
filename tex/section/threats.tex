\section{妥当性の脅威\label{threats}}

\subsection{レコードのみに着目する調査の妥当性}

今回の調査は,レコードのみに着目して議論をおこなっている.
その妥当性について検証するため,以下の条件のもとクラスの総数を調査した.

\begin{itemize}
  \item[(1) ] レコードに必須な前提条件を満たしている.(クラスの修飾子にabstract,sealed,non-sealedのいずれも含まれておらず,継承元のクラスが無い上に,インスタンスイニシャライザも定義されていない.)
  \item[(2) ] (1)に加え,クラスがfinalでないインスタンスフィールドをもたない.
  \item[(3) ] (2)に加え,クラスが1つ以上のfinalなインスタンスフィールドをもつ.
  \item[(4) ] (3)に加え,レコードに変換することで,削減可能な要素が含まれている.(クラスが,カノニカルコンストラクタ,hashCode/equals/toStringメソッド,ゲッタメソッドのいずれかをもつ.)
  \item[(5) ] (4)に加え,クラスがfinalであり,明示的に継承禁止である.
\end{itemize}

ここでいうカノニカルコンストラクタは,引数の型の組み合わせがfinalなインスタンスフィールドと等しいものとしている.
なお,可変長引数の型は配列型として扱っている.

\begin{table}[t]
  \caption{条件別のクラス総数(レコードの使用履歴があったレポジトリ)}
  \label{class_immutability_71}
  \centering
  \begin{tabular}{c||r|r}
      \hline
      条件 & 削減件数 & 割合\\
      \hline\hline
      クラス総数 & 376,672 & -\\
      (1)前提条件を満たすクラス & 182,608 & 48.5\%\\
      (2)finalでないフィールドをもたないクラス & 129,537 & 34.4\%\\
      (3)finalなフィールドをもつクラス & 25,814 & 6.9\%\\
      (4)削減可能な要素をもつクラス & 17,685 & 4.7\%\\
      (5)明示的に継承禁止のクラス & 3,779 & 1.0\%\\
      \hline
  \end{tabular}
\end{table}

\begin{table}[t]
  \caption{条件別のクラス総数(レコードの使用履歴がなかったレポジトリ)}
  \label{class_immutability_1929}
  \centering
  \begin{tabular}{c||r|r}
      \hline
      条件 & 削減件数 & 割合\\
      \hline\hline
      クラス総数 & 1,278,418 & -\\
      (1)前提条件を満たすクラス & 540,517 & 42.3\%\\
      (2)finalでないフィールドをもたないクラス & 321,313 & 25.1\%\\
      (3)finalなフィールドをもつクラス & 93,959 & 7.3\%\\
      (4)削減可能な要素をもつクラス & 62,775 & 4.9\%\\
      (5)明示的に継承禁止のクラス & 13,002 & 1.0\%\\
      \hline
  \end{tabular}
\end{table}

\tabref{class_immutability_71}は,\ref{rq1_method}節の調査でレコードの使用履歴があった71レポジトリにおける結果である.
また,\tabref{class_immutability_1929}はそれ以外の1,929のレポジトリにおける結果である.
これらの結果から,クラスをレコードへのリファクタリングが適格なもの(条件(4)以降を満たすもの)のみに限ったとしても,レコードの検出数である3,244件は十分に大きな値ではないといえる.
すなわち,本研究の結果やそれに基づいた知見は,今後これらのクラスのレコードへの置換が進んだ場合などに,大きく変化する可能性がある.

\subsection{型の同定について}
\ref{field_type_trend}節,\ref{interface_trend}節などでは,型の情報を取得する調査を行っている.
本研究での型の情報は,参照ではなく文字列ベースで取得しているため,同じ型にラベル付けされていても等しい型とは限らない.
実際,\tabref{class_interface_types}のActionListerは,java.awt.eventパッケージのものとユーザ定義のものが取得されている.

\subsection{目視調査について}
本研究での目視調査は,第一著者のみによるものである.
今後は妥当性の保証のため,複数人での目視調査が望まれる.