\section{はじめに\label{intro}}

今日広く利用されているプログラミング言語であるJavaは,円滑なコーディングを目的として新たな言語仕様を導入することがある\cite{Usage_Of_Java_Language_Feature}.
過去のJava言語アップデートでもジェネリクスやラムダ式といった言語仕様が追加されており,これらが開発者たちによってどのように利用されているのか,そしてコーディング上でどのような恩恵をもたらすのか調査が行われてきた\cite{Generics_Research}\cite{Lambda_Research}.

2021年3月のアップデートであるJava16で,Javaはレコード・クラス(以下レコードと記述)という新たな言語仕様を導入した.
レコードはイミュータブルな値ベース・クラスの宣言を簡潔に行うことを可能とし,ソースコード記述量の削減に寄与すると考えられている.
しかしながら,レコードに関して,開発者たちによる利用やコーディング上の恩恵を報告する研究はまだ行われていない.

そこで本稿ではレコードの利用に関する初期調査として,GitHub上のOSSにおけるレコードの利用実態を評価した.
本稿における調査の目的として,次のようなものを掲げている.
\begin{quote}
  \begin{itemize}
    \item レコードについて利用実態に関する各種統計を与える.
    \item Java言語を利用する開発者に対し,レコードの効果的な利用のヒントを与える.
    \item レコードを用いたリファクタリングの支援を行うツール開発にあたり,必要な知見を収集する.
  \end{itemize}
\end{quote}

そして,上記の目的に基づき,次の4つのRQを設定している.
\begin{itemize}
  \item[RQ1 : ] レコードはOSSにおいて,どの程度の数使用されているのか?
  \item[RQ2 : ] 使用されているレコードの特徴はどのようになっているか?
  \item[RQ3 : ] クラスをレコードに変更するリファクタリングでは,どの程度の恩恵を享受できるのか?
  \item[RQ4 : ] クラスからレコードへのリファクタリングを阻害する要因は何か?
\end{itemize}

本稿では,\ref{motivation}節でレコードの仕様を含めた背景と,動機について述べる.
\ref{methodology}節でデータセットと手法について説明し,\ref{result}節で各RQに対する結果を示し,\ref{discussion}節で考察を述べる.
最後に\ref{conclusion}節でまとめを述べる.